\documentclass{beamer}

\usepackage{tikz}
\usetikzlibrary{patterns}
\usetikzlibrary{calc}
\usetikzlibrary{shapes, arrows, positioning}

\usepackage{tabularray}
    \UseTblrLibrary{booktabs}
    \DefTblrTemplate{contfoot-text}{normal}{\lang{(Continued on next page)}{(Continua na próxima página)}} 
    \SetTblrTemplate{contfoot-text}{normal} 
    \DefTblrTemplate{conthead-text}{normal}{\lang{(continued)}{(continuação)}} 
    \SetTblrTemplate{conthead-text}{normal}

\usepackage{enumitem}
\definecolor{ifscgreen}{RGB}{50,93,61}

\setlist[enumerate,1]{label=\colorbox{ifscgreen}{\textcolor{white}{\arabic*}}\quad, leftmargin=*}
\setlist[enumerate,2]{label=\colorbox{ifscgreen!70}{\textcolor{white}{\arabic{enumi}.\arabic*}}\quad, leftmargin=*}
\setlist[enumerate,3]{label=\colorbox{ifscgreen!40}{\textcolor{white}{\arabic{enumi}.\arabic{enumii}.\arabic*}}\quad, leftmargin=*}
% ------------------------------
% Codificação e idioma
% ------------------------------
% \usepackage[utf8]{inputenc}
\usepackage[T1]{fontenc}
\usepackage[brazil]{babel}

% ------------------------------
% Matemática e símbolos
% ------------------------------
\usepackage{amsmath}

% ------------------------------
% Gráficos e figuras
% ------------------------------
\usepackage{graphicx}
\usepackage{subfigure}

% ------------------------------
% Cores, URL e hiperlinks
% ------------------------------
\usepackage{url,color}
\usepackage{hyperref}
\hypersetup{
    pdfstartview={Fit},
    pdftitle={\@title},
    pdfsubject={Engenharia de Telecomunicacoes - IFSC},
    pdfauthor={\@author}
}

% ------------------------------
% Listagens e pseudocódigo
% ------------------------------
\usepackage[
  plainruled,
  noline
]{algorithm2e}

% ------------------------------
% Bibliografia
% ------------------------------
\usepackage[backend=biber,style=numeric,citestyle=ieee]{biblatex}
\addbibresource{references.bib}
\setbeamertemplate{bibliography item}{\insertbiblabel}

% ------------------------------
% Outros pacotes
% ------------------------------
\usepackage{ifsc} % (pacote personalizado, ok manter)

% ------------------------------
% Macros e comandos personalizados
% ------------------------------
\newcommand{\tab}[1]{\hspace{.2\textwidth}\rlap{#1}}

\newcommand{\DAY}{\the\day}
\newcommand{\MONTH}{%
    \ifcase\the\month
    \or Janeiro%
    \or Fevereiro%
    \or Março%
    \or Abril%
    \or Maio%
    \or Junho%
    \or Julho%
    \or Agosto%
    \or Setembro%
    \or Outubro%
    \or Novembro%
    \or Dezembro%
    \fi}
\newcommand{\YEAR}{\the\year}

% ------------------------------
% Dados do título
% ------------------------------
\title{PRG203402 - Lógica de Programação:}
\subtitle{\LARGE Algoritmos e Métodos para Representá-los}
\author{João Cláudio Elsen Barcellos}
\date{\scriptsize \DAY~de \MONTH~de \YEAR}
\institute{
    Engenheiro Eletricista\\
    Formado na Universidade Federal de Santa Catarina\\
    campus Florianópolis\\
    \url{joaoclaudiobarcellos@gmail.com}
}

% ------------------------------
% Customização das seções no Beamer
% ------------------------------
\def\sectionname{}
\def\insertsectionnumber{}
\def\subsectionname{}
\def\insertsubsectionnumber{}

\AtBeginSection[]{
  \begin{frame}
      \vfill
      \centering
      \begin{beamercolorbox}[sep=8pt,center,shadow=true,rounded=true]{title}
          \usebeamerfont{title}\insertsectionhead\par%
      \end{beamercolorbox}
      \vfill
  \end{frame}
}

\setbeamertemplate{caption}[numbered]

% ------------------------------
% Controle de idioma
% ------------------------------
\usepackage{ifthen}
\newboolean{english}
\setboolean{english}{false} % true = inglês, false = português

% Configuração de idioma e títulos
\ifthenelse{\boolean{english}}{
  \usepackage[english]{babel}
  \renewcommand{\figurename}{Figure}
  \renewcommand{\tablename}{Table}
  \SetAlgorithmName{Algorithm}{}{}

  \SetKwInput{KwData}{Input}
  \SetKwInput{KwResult}{Output}
  \SetKwIF{If}{ElseIf}{Else}{if}{then}{else if}{else}{end if}
  \SetKwFor{While}{while}{do}{end while}
  \SetKwFor{For}{for}{do}{end for}
  \SetKw{Return}{return}
}{
  \usepackage[brazil]{babel}
  \renewcommand{\figurename}{Figura}
  \renewcommand{\tablename}{Tabela}
  \SetAlgorithmName{Algoritmo}{}{}

  \SetKwInput{KwData}{Entrada}
  \SetKwInput{KwResult}{Saída}
  \SetKwIF{If}{ElseIf}{Else}{se}{então}{senão se}{senão}{fim}
  \SetKwFor{While}{enquanto}{faça}{fim}
  \SetKwFor{For}{para}{faça}{fim}
  \SetKw{Return}{retorna}
}
% Comando para alternar entre idiomas (inglês/português)
\newcommand{\lang}[2]{\ifbool{english}{#1}{#2}}

\begin{document}

\captionsetup{labelformat=empty}

\begin{frame}[t]
    % Example of a note:
    % \note{Boa tarde a todos, na aula de hoje iremos estudar os circuitos sequenciais...}
    \maketitle
    \vspace{-1cm}
    \begin{flushleft}
        \vfill
        \textit{\tiny $^{*}$Créditos ao Prof. Emerson Ribeiro de Mello, o qual criou e disponibilizou o template aqui usado, via ShareLaTeX}\par
        \textit{\tiny $^{**}$Créditos ao Prof. Renan Augusto Starke, o qual forneceu parte do conteúdo usado nesta apresentação}
    \end{flushleft}
\end{frame}

\begin{frame}[t]{Na aula de hoje veremos...}
    \tableofcontents
\end{frame}

\section{Lógica de programação e algoritmos}
\begin{frame}{Lógica de programação e algoritmos}
    \begin{alertblock}{Lógica de programação:}
        A técnica de \textbf{encadear pensamentos/instruções} para atingir determinado objetivo.
    \end{alertblock}   
    \begin{alertblock}{Algoritmo:}
        Formalmente, é uma sequência finita de passos que levam à execução de uma tarefa. 
    \end{alertblock}
\end{frame}

\section{Algoritmos}
\begin{frame}{Algoritmos}
    
    \small 
    
    Tarefas cotidianas:

    \begin{enumerate}
        \item Fazer um sanduíche;
        \item Trocar uma lâmpada;
        \item Receitas...
    \end{enumerate}

    Problemas mais complexos e/ou que podem ser automatizados:

    \begin{enumerate}
        \item Cálculo da sequencia de Fibonacci;
        \item Ordenação de números;
        \item Resolução numérica de equações diferenciais;
        \item Solução de circuitos;
        \item Entre muitas outras...
    \end{enumerate} 
\end{frame}

\begin{frame}{Estruturas de decisão}
    
    \small

    Fazer café:

    \begin{enumerate}
        \item Pegue a cafeteira
        \item Se cafeteira suja:
        \begin{enumerate}
            \item Limpar cafeteira
        \end{enumerate}
        \item Pegue o pó
        \item Pegue o filtro
        \item Complete a cafeteira com água até o nível permitido
        \item Coloque 2 colheres de pó
        \item Ligue a cafeteira
        \item Beba!
    \end{enumerate}
\end{frame}

\begin{frame}{Estruturas de repetição e decisão}
    
    \small

    Fazer café (na cafeteria):

    \begin{enumerate}
        \item Aguardando novo cliente
        \item Se houver café pronto:
        \begin{enumerate}
            \item Servir cliente!
            \item Ir para passo 1
        \end{enumerate}
        \item Senão:
        \begin{enumerate}
            \item Se cafeteira suja:
            \begin{enumerate}
                \item Limpar cafeteira
            \end{enumerate}
            \item Pegue pó
            \item Pegue o filtro
            \item Complete a cafeteira com água até o nível permitido
            \item Coloque 2 colheres de pó
            \item Ligue a cafeteira
            \item Aguardar
            \item Ir para passo 2.1
        \end{enumerate}
    \end{enumerate}
\end{frame}

\section{Métodos de representação de algoritmos}
\begin{frame}{Métodos de representação de algoritmos}
    
    \small

    Há diversas formas de representar um algorítmo:

    \begin{enumerate}
        \item ``Linguagem natural'' (e.g., por meio de descrição, similar ao exemplo do café);
        \item Fluxogramas;
        \item Diagramas de Chapin (Nassi–Shneiderman);
        \item Pseudocódigo.
    \end{enumerate}
\end{frame}

\subsection{Fluxograma}
\begin{frame}[fragile]{Principais símbolos de fluxograma}
    \centering
    \makebox[\linewidth]{% <-- centraliza horizontalmente a tabela
      \resizebox{0.8\linewidth}{!}{%
        \begin{tblr}[
          caption = {Principais símbolos usados em fluxogramas}, 
          label = {tab:fluxo_simbolos}
          ]
          {
          colspec = {X[c,m,1.2] X[l,m,2.5] X[l,m,4]},
          cells = {font = \fontsize{9pt}{11pt}\selectfont},
          rowhead = 1
          }
          \toprule
          \textbf{Símbolo} & \textbf{Nome} & \textbf{Descrição} \\
          \midrule
          \tikz\node[draw, rectangle, minimum width=1cm, minimum height=0.7cm, fill=blue!20]{}; & Processo & Representa uma instrução ou ação a ser executada \\
          \tikz\node[draw, ellipse, minimum width=1cm, minimum height=0.7cm, fill=green!20]{}; & Início / Fim & Indica o ponto de entrada ou término do algoritmo \\
          \tikz\node[draw, diamond, aspect=2, minimum width=1cm, minimum height=0.7cm, fill=blue!20]{}; & Decisão & Representa uma escolha condicional (ex: if, while) \\
          \tikz\node[draw, trapezium, trapezium left angle=70, trapezium right angle=110, minimum width=1cm, minimum height=0.7cm, fill=blue!20]{}; & Entrada / Saída & Indica leitura ou escrita de dados \\
          \tikz\draw[->, thick, color=black] (0,0) -- (0.5,0); & Seta / Fluxo & Mostra o fluxo de execução entre os blocos \\
          \bottomrule
        \end{tblr}%
      }%
    }
\end{frame}

\begin{frame}[fragile]{Exemplo: Fazer café}
    \centering
    \begin{figure}[!ht]
        \centering
        \label{fig:fluxo_cafe}
        \resizebox{0.4\textwidth}{!}{
          \begin{tikzpicture}[
              node distance=1.2cm,
              every node/.style={font=\scriptsize},
              align=center
              ]
              \node (start) [draw, ellipse, fill=green!20, minimum width=1.2cm, minimum height=0.8cm] {Início};
              \node (check) [draw, diamond, below of=start, fill=blue!20, inner sep=0cm, yshift=-0.4cm] {Cafeteira\\limpa?};
              \node (clean) [draw, rectangle, right of=check, xshift=3cm, fill=blue!20, minimum width=1.2cm, minimum height=0.8cm] {Limpar\\cafeteira};
              \node (fill) [draw, rectangle, below of=check, yshift=-0.8cm, fill=blue!20, minimum width=1.2cm, minimum height=0.8cm] {Adicionar\\água e pó};
              \node (on) [draw, rectangle, below of=fill, fill=blue!20, minimum width=1.2cm, minimum height=0.8cm] {Ligar\\cafeteira};
              \node (drink) [draw, rectangle, below of=on, fill=blue!20, minimum width=1.2cm, minimum height=0.8cm] {Beber\\café};
              \node (end) [draw, ellipse, below of=drink, fill=green!20, minimum width=1.2cm, minimum height=0.8cm] {Fim};

              \draw[->] (start) -- (check);
              \draw[->] (check) -- node[above] {não} (clean);
              \draw[->] (clean) |- (fill);
              \draw[->] (check) -- node[left] {sim} (fill);
              \draw[->] (fill) -- (on);
              \draw[->] (on) -- (drink);
              \draw[->] (drink) -- (end);
          \end{tikzpicture}
        }
    \end{figure}
\end{frame}

\begin{frame}[fragile]{Exemplo: Trocar lâmpada}
    \centering
    \begin{figure}[!ht]
        \centering
        \label{fig:fluxo_lampada}
        \resizebox{0.65\textwidth}{!}{
          \begin{tikzpicture}[
              node distance=1.2cm,
              every node/.style={font=\scriptsize},
              align=center
              ]
              \node (start) [draw, ellipse, fill=green!20, minimum width=1.2cm, minimum height=0.8cm] {Início};
              \node (switch) [draw, rectangle, below of=start, fill=blue!20, minimum width=1.2cm, minimum height=0.8cm] {Acionar\\interruptor};
              \node (light) [draw, diamond, below of=switch, fill=blue!20, inner sep=0cm, yshift=-0.4cm] {Lâmpada\\acendeu?};
              \node (end) [draw, ellipse, right of=light, xshift=3.5cm, fill=green!20, minimum width=1.2cm, minimum height=0.8cm] {Fim};
              \node (replace) [draw, rectangle, below of=light, yshift=-0.8cm, fill=blue!20, minimum width=1.2cm, minimum height=0.8cm] {Trocar\\lâmpada};

              \draw[->] (start) -- (switch);
              \draw[->] (switch) -- (light);
              \draw[->] (light) -- node[above] {sim} (end);
              \draw[->] (light) -- node[left] {não} (replace);
              \draw[->] (replace.west) -- ++(-1,0) |- (switch.west);
          \end{tikzpicture}
        }
    \end{figure}
\end{frame}

\begin{frame}[fragile]{Exemplo: Sair de casa com guarda-chuva}
    \centering
    \begin{figure}[!ht]
        \centering
        \label{fig:fluxo_guarda_chuva}
        \resizebox{0.5\textwidth}{!}{
          \begin{tikzpicture}[
              node distance=1.2cm,
              every node/.style={font=\scriptsize},
              align=center
              ]
              \node (start) [draw, ellipse, fill=green!20, minimum width=1.2cm, minimum height=0.8cm] {Início};
              \node (check) [draw, diamond, below of=start, fill=blue!20, inner sep=0cm, yshift=-0.4cm] {Vai\\chover?};
              \node (take) [draw, rectangle, right of=check, xshift=3cm, fill=blue!20, minimum width=1.2cm, minimum height=0.8cm] {Pegar\\guarda-chuva};
              \node (go) [draw, rectangle, below of=check, yshift=-0.8cm, fill=blue!20, minimum width=1.2cm, minimum height=0.8cm] {Sair\\de casa};
              \node (end) [draw, ellipse, below of=go, fill=green!20, minimum width=1.2cm, minimum height=0.8cm] {Fim};

              \draw[->] (start) -- (check);
              \draw[->] (check) -- node[above] {sim} (take);
              \draw[->] (take) |- (go);
              \draw[->] (check) -- node[left] {não} (go);
              \draw[->] (go) -- (end);
          \end{tikzpicture}
        }
    \end{figure}
\end{frame}

\begin{frame}[fragile]{Exemplo: Cálculo de média e aprovação}
    \centering
    \begin{figure}[!ht]
        \centering
        \label{fig:fluxo_media_aprovacao}
        \resizebox{0.8\textwidth}{!}{
          \begin{tikzpicture}[
              node distance=1.2cm and 2cm,
              every node/.style={font=\scriptsize},
              align=center
              ]
              \node (start) [draw, ellipse, fill=green!20, minimum width=1.2cm, minimum height=0.8cm] {Início};
              \node (read) [draw, trapezium, trapezium left angle=70, trapezium right angle=110, below of=start, fill=blue!20, minimum width=1.2cm, minimum height=0.8cm] {Leia\\N1, N2};
              \node (calc) [draw, rectangle, below of=read, fill=blue!20, minimum width=1.2cm, minimum height=0.8cm] {media =\\(N1 + N2)/2};
              \node (check) [draw, diamond, below of=calc, fill=blue!20, inner sep=0cm, yshift=-0.4cm] {media $\geq$\\6?};
              \node (approved) [draw, trapezium, trapezium left angle=70, trapezium right angle=110, right of=check, xshift=3cm, fill=blue!20, minimum width=1.2cm, minimum height=0.8cm] {Aprovado!};
              \node (recovery) [draw, trapezium, trapezium left angle=70, trapezium right angle=110, left of=check, xshift=-3cm, fill=blue!20, minimum width=1.2cm, minimum height=0.8cm] {Recuperação!};
              \node (end) [draw, ellipse, below of=check, yshift=-2.5cm, fill=green!20, minimum width=1.2cm, minimum height=0.8cm] {Fim};

              \draw[->] (start) -- (read);
              \draw[->] (read) -- (calc);
              \draw[->] (calc) -- (check);
              \draw[->] (check) -- node[above] {sim} (approved);
              \draw[->] (check) -- node[above] {não} (recovery);
              \draw[->] (approved) |- (end);
              \draw[->] (recovery) |- (end);
          \end{tikzpicture}
        }
    \end{figure}
\end{frame}

\subsection{Pseudocódigo}
\begin{frame}[fragile]{Pseudocódigo}

    \SetAlgoNlRelativeSize{0}
    \SetNlSty{textbf}{}{}
    \SetAlgoNoLine
    \SetAlgoNoEnd
    \renewcommand{\thealgocf}{}

    \centering
    \makebox[\linewidth]{%
      \resizebox{0.8\linewidth}{!}{%
        \begin{minipage}{\textwidth}
            \begin{algorithm}[H]
                \KwData{Lâmpada e interruptor}
                \KwResult{Lâmpada funcionando}
                Acione o interruptor\;
                \If{a lâmpada não acendeu}{
                  Buscar uma nova lâmpada\;
                  \While{a lâmpada não acender}{
                    Retirar a lâmpada queimada\;
                    Colocar a nova lâmpada\;
                  }
                }
                \Return Lâmpada funcionando\;
            \end{algorithm}
        \end{minipage}
      }
    }
\end{frame}

\begin{frame}{Boas práticas}
    \small

    \begin{enumerate}
        \item Representação dos algoritmos fazem parte da documentação do projeto: deveria-se mantê-los sempre atualizados (boas práticas de engenharia de software);
        \item Usar referências, como \cite{forbellone2005}, para consultar ao longo do desenvolvimento do seu projeto; 
    \end{enumerate}
\end{frame}

% \section{Exercícios}


% \frame{ 
% \frametitle{Exercícios}

% \vfill Construa o fluxograma e o pseudocódigo para (use \url{https://creately.com/app}):
% 
% \begin{enumerate}
%     \vfill \item Dado um número real qualquer, informe qual é o seu dobro.
%     \vfill \item Dadas as medidas de uma sala em metros (comprimento e largura), informe a sua área em metros quadrados.
%     \vfill \item Dados um salário e um percentual de reajuste, calcule o salário reajustado. Considere que o percentual de reajuste é dado por um número real entre 0 e 1. Por exemplo, se o reajuste for de 15\%, o usuário deve digitar o número 0.15.
%     \vfill \item Dados o valor da compra e o percentual de desconto, calcule o valor a ser pago. Considere que o percentual de desconto é um número real entre 0 e 1. 
% \end{enumerate}

% }
%   
%   \frame{ 
%   \frametitle{Exercícios}
%   
%   \vfill Construa um fluxograma que:
%   
%   \begin{itemize}
%       \item Leia a cotação do dólar
%       \item Leia um valor em dólares
%       \item Converta esse valor para Real
%       \item Mostre o resultado
%   \end{itemize}
%   
%   \vfill Desenvolva um fluxograma que:
%   
%   \begin{itemize}
%       
%       \item Leia 4 (quatro) números
%       \item Calcule o quadrado para cada um
%       \item Somem todos e
%       \item Mostre o resultado
%   \end{itemize}
% }

%   \begin{frame}
%       \frametitle{Exercícios}
%       
%       Construa um algoritmo para pagamento de comissão de vendedores de peças, levando-se
%       em consideração que sua comissão será de 5\% do total da venda e que você tem seguintes dados:
%       
%       \begin{itemize}
%           \item Identificação do vendedor
%           \item Código da peça
%           \item Preço unitário da peça
%           \item Quantidade vendida
%       \end{itemize}
%       
%       Construa o diagrama de blocos do algoritmo desenvolvido, e por fim faça um teste de mesa.
%       
%       
%       
%   \end{frame}

\section{Referências}
\begin{frame}[t, allowframebreaks]
    \frametitle{References}
    \printbibliography
\end{frame}

% \section{Referências}
% \begin{frame}{Referências}   
%     \begin{itemize}
%         \item \vfill \href{https://arcade-book.readthedocs.io}{Python Arcade}
%         \item \vfill \href{http://www.pymunk.org/}{Python Pymunk}
%         \item \vfill FORBELLONE, A. L. V. \textbf{Lógica de Programação: a construção de algoritmos e estruturas de dados.} 3.ed. São Paulo: Prentice Hall, 2005.
%         \item \vfill  Eric Freeman. \textbf{Head First Learn to Code}. O'Reilly, 2018.
%     \end{itemize}
% \end{frame}

\end{document}

%
%
% Local variables section, for Emacs: specifies that this file is the main one.
%
%
%%% Local Variables:
%%% TeX-master: t
%%% End:
