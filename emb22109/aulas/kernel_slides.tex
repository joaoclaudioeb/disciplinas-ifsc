\documentclass{beamer}

\usepackage{tikz}
\usetikzlibrary{patterns}
\usetikzlibrary{calc}
\usetikzlibrary{shapes, arrows, positioning}

\usepackage{tabularray}
\UseTblrLibrary{booktabs}
\DefTblrTemplate{contfoot-text}{normal}{\lang{(Continued on next page)}{(Continua na próxima página)}} 
\SetTblrTemplate{contfoot-text}{normal} 
\DefTblrTemplate{conthead-text}{normal}{\lang{(continued)}{(continuação)}} 
\SetTblrTemplate{conthead-text}{normal}

\usepackage{enumitem}
\definecolor{ifscgreen}{RGB}{50,93,61}

\setlist[enumerate,1]{label=\colorbox{ifscgreen}{\textcolor{white}{\arabic*}}\quad, leftmargin=*}
\setlist[enumerate,2]{label=\colorbox{ifscgreen!70}{\textcolor{white}{\arabic{enumi}.\arabic*}}\quad, leftmargin=*}
\setlist[enumerate,3]{label=\colorbox{ifscgreen!40}{\textcolor{white}{\arabic{enumi}.\arabic{enumii}.\arabic*}}\quad, leftmargin=*}
% ------------------------------
% Codificação e idioma
% ------------------------------
% \usepackage[utf8]{inputenc}
\usepackage[T1]{fontenc}
\usepackage[brazil]{babel}

% ------------------------------
% Matemática e símbolos
% ------------------------------
\usepackage{amsmath}

% ------------------------------
% Gráficos e figuras
% ------------------------------
\usepackage{graphicx}
\usepackage{subfigure}

% ------------------------------
% Cores, URL e hiperlinks
% ------------------------------
\usepackage{url,color}
\usepackage{hyperref}
\hypersetup{
  pdfstartview={Fit},
  pdftitle={\@title},
  pdfsubject={Engenharia de Telecomunicacoes - IFSC},
  pdfauthor={\@author}
}

% ------------------------------
% Listagens e pseudocódigo
% ------------------------------
\usepackage[
plainruled,
noline
]{algorithm2e}

% ------------------------------
% Bibliografia
% ------------------------------
\usepackage[backend=biber,style=numeric,citestyle=ieee]{biblatex}
\addbibresource{references.bib}
\setbeamertemplate{bibliography item}{\insertbiblabel}

% ------------------------------
% Outros pacotes
% ------------------------------
\usepackage{../../latex/ifsc_slides} % (pacote personalizado, ok manter)

% ------------------------------
% Macros e comandos personalizados
% ------------------------------
\newcommand{\tab}[1]{\hspace{.2\textwidth}\rlap{#1}}

\newcommand{\DAY}{\the\day}
\newcommand{\MONTH}{%
  \ifcase\the\month
  \or Janeiro%
  \or Fevereiro%
  \or Março%
  \or Abril%
  \or Maio%
  \or Junho%
  \or Julho%
  \or Agosto%
  \or Setembro%
  \or Outubro%
  \or Novembro%
  \or Dezembro%
  \fi}
\newcommand{\YEAR}{\the\year}

% ------------------------------
% Dados do título
% ------------------------------
\title{EMB22109 - Sistemas Embarcados:}
\subtitle{\LARGE Introdução a Sistemas Operacionais e Tipos de Kernel}
\author{Hugo Marcondes}
\date{\scriptsize \DAY~de \MONTH~de \YEAR}
\institute{
  Instituto Federal de Educação, Ciência e Tecnologia de Santa Catarina\\
  Departamento Acadêmico de Eletrônica\\
  \url{hugo.marcondes@ifsc.edu.br}
}

% ------------------------------
% Customização das seções no Beamer
% ------------------------------
\def\sectionname{}
\def\insertsectionnumber{}
\def\subsectionname{}
\def\insertsubsectionnumber{}

\AtBeginSection[]{
  \begin{frame}
      \vfill
      \centering
      \begin{beamercolorbox}[sep=8pt,center,shadow=true,rounded=true]{title}
          \usebeamerfont{title}\insertsectionhead\par%
      \end{beamercolorbox}
      \vfill
  \end{frame}
}

\setbeamertemplate{caption}[numbered]

% ------------------------------
% Controle de idioma
% ------------------------------
\usepackage{ifthen}
\newboolean{english}
\setboolean{english}{false} % true = inglês, false = português

% Configuração de idioma e títulos
\ifthenelse{\boolean{english}}{
  \usepackage[english]{babel}
  \renewcommand{\figurename}{Figure}
  \renewcommand{\tablename}{Table}
  \SetAlgorithmName{Algorithm}{}{}
  
  \SetKwInput{KwData}{Input}
  \SetKwInput{KwResult}{Output}
  \SetKwIF{If}{ElseIf}{Else}{if}{then}{else if}{else}{end if}
  \SetKwFor{While}{while}{do}{end while}
  \SetKwFor{For}{for}{do}{end for}
  \SetKw{Return}{return}
}{
  \usepackage[brazil]{babel}
  \renewcommand{\figurename}{Figura}
  \renewcommand{\tablename}{Tabela}
  \SetAlgorithmName{Algoritmo}{}{}
  
  \SetKwInput{KwData}{Entrada}
  \SetKwInput{KwResult}{Saída}
  \SetKwIF{If}{ElseIf}{Else}{se}{então}{senão se}{senão}{fim}
  \SetKwFor{While}{enquanto}{faça}{fim}
  \SetKwFor{For}{para}{faça}{fim}
  \SetKw{Return}{retorna}
}
% Comando para alternar entre idiomas (inglês/português)
\newcommand{\lang}[2]{\ifbool{english}{#1}{#2}}
    
% Default fixed font does not support bold face
\DeclareFixedFont{\ttb}{T1}{txtt}{bx}{n}{8} % for bold
\DeclareFixedFont{\ttm}{T1}{txtt}{m}{n}{8}  % for normal

% Custom colors
\usepackage{color}
\definecolor{deepblue}{rgb}{0,0,0.5}
\definecolor{deepred}{rgb}{0.6,0,0}
\definecolor{deepgreen}{rgb}{0,0.5,0}

\usepackage{listings}

% Python style for highlighting
\newcommand\pythonstyle{\lstset{
language=Python,
basicstyle=\ttm\tiny,
morekeywords={self},              % Add keywords here
keywordstyle=\ttb\color{deepblue},
emph={MyClass,__init__},          % Custom highlighting
emphstyle=\ttb\color{deepred},    % Custom highlighting style
stringstyle=\color{deepgreen},
frame=tb,                         % Any extra options here
showstringspaces=false,
numbers=none 
}}


% Python environment
\lstnewenvironment{python}[1][]
{
\pythonstyle
\lstset{#1}
}
{}

% Python for external files
\newcommand\pythonexternal[2][]{{
\pythonstyle
\lstinputlisting[#1]{#2}}}

% Python for inline
\newcommand\pythoninline[1]{{\pythonstyle\lstinline!#1!}}    
    
    \lstset{
      inputencoding = utf8,  % Input encoding
      extendedchars = true,  % Extended ASCII
      literate      =        % Support additional characters
      {á}{{\'a}}1  {é}{{\'e}}1  {í}{{\'i}}1 {ó}{{\'o}}1  {ú}{{\'u}}1
      {Á}{{\'A}}1  {É}{{\'E}}1  {Í}{{\'I}}1 {Ó}{{\'O}}1  {Ú}{{\'U}}1
      {à}{{\`a}}1  {è}{{\`e}}1  {ì}{{\`i}}1 {ò}{{\`o}}1  {ù}{{\`u}}1
      {À}{{\`A}}1  {È}{{\`E}}1  {Ì}{{\`I}}1 {Ò}{{\`O}}1  {Ù}{{\`U}}1
      {ä}{{\"a}}1  {ë}{{\"e}}1  {ï}{{\"i}}1 {ö}{{\"o}}1  {ü}{{\"u}}1
      {Ä}{{\"A}}1  {Ë}{{\"E}}1  {Ï}{{\"I}}1 {Ö}{{\"O}}1  {Ü}{{\"U}}1
      {â}{{\^a}}1  {ê}{{\^e}}1  {î}{{\^i}}1 {ô}{{\^o}}1  {û}{{\^u}}1
      {Â}{{\^A}}1  {Ê}{{\^E}}1  {Î}{{\^I}}1 {Ô}{{\^O}}1  {Û}{{\^U}}1
      {œ}{{\oe}}1  {Œ}{{\OE}}1  {æ}{{\ae}}1 {Æ}{{\AE}}1  {ß}{{\ss}}1
      {ẞ}{{\SS}}1  {ç}{{\c{c}}}1 {Ç}{{\c{C}}}1 {ø}{{\o}}1  {Ø}{{\O}}1
      {å}{{\aa}}1  {Å}{{\AA}}1  {ã}{{\~a}}1  {õ}{{\~o}}1 {Ã}{{\~A}}1
      {Õ}{{\~O}}1  {ñ}{{\~n}}1  {Ñ}{{\~N}}1  {¿}{{?`}}1  {¡}{{!`}}1
      {„}{\quotedblbase}1 {"}{\textquotedblleft}1 {–}{$-$}1
      {°}{{\textdegree}}1 {º}{{\textordmasculine}}1 {ª}{{\textordfeminine}}1
      {£}{{\pounds}}1  {©}{{\copyright}}1  {®}{{\textregistered}}1
      {«}{{\guillemotleft}}1  {»}{{\guillemotright}}1  {Ð}{{\DH}}1  {ð}{{\dh}}1
      {Ý}{{\'Y}}1    {ý}{{\'y}}1    {Þ}{{\TH}}1    {þ}{{\th}}1    {Ă}{{\u{A}}}1
      {ă}{{\u{a}}}1  {Ą}{{\k{A}}}1  {ą}{{\k{a}}}1  {Ć}{{\'C}}1    {ć}{{\'c}}1
      {Č}{{\v{C}}}1  {č}{{\v{c}}}1  {Ď}{{\v{D}}}1  {ď}{{\v{d}}}1  {Đ}{{\DJ}}1
      {đ}{{\dj}}1    {Ė}{{\.{E}}}1  {ė}{{\.{e}}}1  {Ę}{{\k{E}}}1  {ę}{{\k{e}}}1
      {Ě}{{\v{E}}}1  {ě}{{\v{e}}}1  {Ğ}{{\u{G}}}1  {ğ}{{\u{g}}}1  {Ĩ}{{\~I}}1
      {ĩ}{{\~\i}}1   {Į}{{\k{I}}}1  {į}{{\k{i}}}1  {İ}{{\.{I}}}1  {ı}{{\i}}1
      {Ĺ}{{\'L}}1    {ĺ}{{\'l}}1    {Ľ}{{\v{L}}}1  {ľ}{{\v{l}}}1  {Ł}{{\L{}}}1
      {ł}{{\l{}}}1   {Ń}{{\'N}}1    {ń}{{\'n}}1    {Ň}{{\v{N}}}1  {ň}{{\v{n}}}1
      {Ő}{{\H{O}}}1  {ő}{{\H{o}}}1  {Ŕ}{{\'{R}}}1  {ŕ}{{\'{r}}}1  {Ř}{{\v{R}}}1
      {ř}{{\v{r}}}1  {Ś}{{\'S}}1    {ś}{{\'s}}1    {Ş}{{\c{S}}}1  {ş}{{\c{s}}}1
      {Š}{{\v{S}}}1  {š}{{\v{s}}}1  {Ť}{{\v{T}}}1  {ť}{{\v{t}}}1  {Ũ}{{\~U}}1
      {ũ}{{\~u}}1    {Ū}{{\={U}}}1  {ū}{{\={u}}}1  {Ů}{{\r{U}}}1  {ů}{{\r{u}}}1
      {Ű}{{\H{U}}}1  {ű}{{\H{u}}}1  {Ų}{{\k{U}}}1  {ų}{{\k{u}}}1  {Ź}{{\'Z}}1
      {ź}{{\'z}}1    {Ż}{{\.Z}}1    {ż}{{\.z}}1    {Ž}{{\v{Z}}}1  {ž}{{\v{z}}}1
      % ¿ and ¡ are not correctly displayed if inconsolata font is used
      % together with the lstlisting environment. Consider typing code in
      % external files and using \lstinputlisting to display them instead.      
    }
    
\begin{document}

\captionsetup{labelformat=empty}

\begin{frame}[t]
    % Example of a note:
    % \note{Boa tarde a todos, na aula de hoje iremos estudar os circuitos sequenciais...}
    \maketitle
    \vspace{-1cm}
    \begin{flushleft}
        \vfill
        \textit{\tiny $^{*}$Créditos ao Prof. Emerson Ribeiro de Mello, o qual criou e disponibilizou o template aqui usado, via ShareLaTeX}\par
    \end{flushleft}
\end{frame}

\begin{frame}[t]{Na aula de hoje veremos...}
    \tableofcontents
\end{frame}

\section{Introdução a Sistemas Operacionais}

\begin{frame}[fragile]
\frametitle{Sistema Operacional}

    \vfill \begin{block}{Definição}
        \textbf{Sistema Operacional} é uma camada de software que gerencia os recursos de hardware e fornece serviços para programas de aplicação.
    \end{block}

    \vfill\begin{center}
     \includegraphics[scale=0.5]{example-image}
     \end{center}
\end{frame}

\begin{frame}[fragile]
\frametitle{Sistema Operacional}

\begin{columns}
    \begin{column}{0.5\textwidth}
        \begin{enumerate}
            \item Visam FACILITAR o uso de sistemas computacionais
            \item Sistemas Computacionais
            \begin{enumerate}
                \item CPU + Memória + I/O
            \end{enumerate}
            \item O sistema operacional pode ser visto sob duas perspectivas:
            \begin{enumerate}
                \item Máquina Virtual
                \item Gerenciador de Recursos
            \end{enumerate}
        \end{enumerate}
    \end{column}
    
    \begin{column}{0.5\textwidth}
        \begin{center}
            \includegraphics[scale=0.4]{example-image}
        \end{center}
    \end{column}
\end{columns}
\end{frame}

\begin{frame}[fragile]
\frametitle{SO como Máquina Virtual}

\begin{enumerate}
    \vfill \item O sistema operacional "estende" o hardware para implementar uma interface de mais alto nível para as aplicações
    \vfill \item Importante característica para prover portabilidade
    \vfill \item Abstrai detalhes complexos do hardware
    \vfill \item Fornece uma interface uniforme para programas de aplicação
\end{enumerate}

\vfill \begin{center}
    \includegraphics[scale=0.4]{example-image}
\end{center}
\end{frame}

\begin{frame}[fragile]
\frametitle{SO como Gerenciador de Recursos}

\begin{columns}
    \begin{column}{0.5\textwidth}
        \begin{enumerate}
            \item Gerencia recursos de hardware:
            \begin{enumerate}
                \item CPU
                \item Memória
                \item Dispositivos de I/O
            \end{enumerate}
            \item Controla a execução de programas
            \item Previne erros e uso indevido
            \item Garante o uso eficiente dos recursos
        \end{enumerate}
    \end{column}
    
    \begin{column}{0.5\textwidth}
        \begin{center}
            \includegraphics[scale=0.35]{example-image}
        \end{center}
    \end{column}
\end{columns}
\end{frame}

\begin{frame}[fragile]
\frametitle{Conceitos de Kernel}

\begin{enumerate}
    \vfill \item \textbf{Kernel = Núcleo}
    \begin{enumerate}
        \item Parte central e fundamental de um programa/algoritmo
    \end{enumerate}
    
    \vfill \item \textbf{Kernel = Modo Kernel (Supervisor)}
    \begin{enumerate}
        \item Parte de um programa que executa em modo Kernel
        \item Modo Kernel -> Espaço/Modo de Execução
        \item Suporte do Processador a diferentes espaços ou níveis de execução (modo Kernel e modo usuário)
    \end{enumerate}
\end{enumerate}
\end{frame}

\begin{frame}[fragile]
\frametitle{Kernel de um Sistema Operacional}

\begin{enumerate}
    \vfill \item Núcleo do Sistema Operacional
    \vfill \item Provê o Gerenciamento do Sistema e Funcionalidades Básicas
    \vfill \item Executa em modo privilegiado (modo kernel)
    \vfill \item Controla acesso direto ao hardware
    \vfill \item Implementa serviços fundamentais:
    \begin{enumerate}
        \item Gerenciamento de memória
        \item Gerenciamento de processos
        \item Gerenciamento de dispositivos
        \item Comunicação entre processos
    \end{enumerate}
\end{enumerate}
\end{frame}

\begin{frame}[fragile]
\frametitle{Ambientes de Execução}

\begin{center}
    \includegraphics[scale=0.5]{example-image}
\end{center}

\begin{enumerate}
    \item Diagramas de camadas de Abstração e Anéis de Privilégio de Execução
    \item Diferentes níveis de acesso ao hardware
    \item Proteção contra operações indevidas
\end{enumerate}
\end{frame}

\section{Tipos de Kernel}

\begin{frame}[fragile]
\frametitle{Abordagens de Design de Kernel}

\begin{enumerate}
    \vfill \item \textbf{Kernel Monolítico}
    \vfill \item \textbf{Microkernel}
    \begin{enumerate}
        \item Nanokernel
    \end{enumerate}
    \vfill \item \textbf{Kernel Híbrido}
    \vfill \item \textbf{Exokernel}
\end{enumerate}

\vfill \begin{center}
    \textit{*Kernel Híbrido é tido como um termo controverso dentre a comunidade científica da área.}
\end{center}
\end{frame}

\begin{frame}[fragile]
\frametitle{Kernel Monolítico}

\begin{columns}
    \begin{column}{0.5\textwidth}
        \begin{enumerate}
            \item Todo o sistema operacional executa em modo kernel
            \item Abordagem tradicional
            \item Exemplos:
            \begin{enumerate}
                \item Unix tradicional
                \item Linux
                \item FreeBSD
            \end{enumerate}
            \item Vantagens:
            \begin{enumerate}
                \item Desempenho
                \item Acesso direto ao hardware
            \end{enumerate}
            \item Desvantagens:
            \begin{enumerate}
                \item Menor segurança
                \item Falhas afetam todo o sistema
                \item Código mais complexo
            \end{enumerate}
        \end{enumerate}
    \end{column}
    
    \begin{column}{0.5\textwidth}
        \begin{center}
            \includegraphics[scale=0.3]{example-image}
        \end{center}
    \end{column}
\end{columns}
\end{frame}

\begin{frame}[fragile]
\frametitle{Microkernel}

\begin{columns}
    \begin{column}{0.5\textwidth}
        \begin{enumerate}
            \item Apenas funções essenciais no kernel
            \item Serviços executam como processos em modo usuário
            \item Comunicação via IPC (Comunicação Entre Processos)
            \item Exemplos:
            \begin{enumerate}
                \item MINIX
                \item QNX
                \item L4
            \end{enumerate}
            \item Vantagens:
            \begin{enumerate}
                \item Maior segurança
                \item Falhas isoladas
                \item Código mais modular
            \end{enumerate}
            \item Desvantagens:
            \begin{enumerate}
                \item Overhead de comunicação
                \item Desempenho potencialmente menor
            \end{enumerate}
        \end{enumerate}
    \end{column}
    
    \begin{column}{0.5\textwidth}
        \begin{center}
            \includegraphics[scale=0.3]{example-image}
        \end{center}
    \end{column}
\end{columns}
\end{frame}

\begin{frame}[fragile]
\frametitle{Conceitos e Objetivos do Microkernel}

\begin{enumerate}
    \vfill \item \textbf{Segurança}
    \begin{enumerate}
        \item Menor código possível executando em modo Kernel
        \item Menos tempo de execução do sistema em modo privilegiado
    \end{enumerate}
    
    \vfill \item \textbf{Separação de Serviços em Servidores}
    \begin{enumerate}
        \item Serviços executam como processos independentes
        \item Isolamento de falhas
    \end{enumerate}
    
    \vfill \item \textbf{Comunicação via IPC}
    \begin{enumerate}
        \item Baseada em troca de mensagens
        \item Mecanismo fundamental para operação do sistema
    \end{enumerate}
\end{enumerate}
\end{frame}

\begin{frame}[fragile]
\frametitle{Kernel Híbrido}

\begin{columns}
    \begin{column}{0.5\textwidth}
        \begin{enumerate}
            \item Combina características de kernel monolítico e microkernel
            \item Alguns serviços no espaço do kernel, outros como processos de usuário
            \item Exemplos:
            \begin{enumerate}
                \item Windows NT (Windows 10/11)
                \item macOS (XNU)
                \item DragonFly BSD
            \end{enumerate}
            \item Vantagens:
            \begin{enumerate}
                \item Equilíbrio entre desempenho e modularidade
                \item Flexibilidade de design
            \end{enumerate}
            \item Desvantagens:
            \begin{enumerate}
                \item Complexidade de implementação
                \item Compromissos de design
            \end{enumerate}
        \end{enumerate}
    \end{column}
    
    \begin{column}{0.5\textwidth}
        \begin{center}
            \includegraphics[scale=0.3]{example-image}
        \end{center}
    \end{column}
\end{columns}
\end{frame}

\begin{frame}[fragile]
\frametitle{Exokernel}

\begin{enumerate}
    \vfill \item Abordagem minimalista e radical
    \vfill \item Kernel fornece apenas proteção e multiplexação de recursos
    \vfill \item Bibliotecas em nível de usuário (LibOS) implementam abstrações tradicionais
    \vfill \item Desenvolvido no MIT nos anos 90
    \vfill \item Vantagens:
    \begin{enumerate}
        \item Máxima flexibilidade para aplicações
        \item Personalização de abstrações
        \item Potencial para alto desempenho
    \end{enumerate}
    \vfill \item Desvantagens:
    \begin{enumerate}
        \item Complexidade para desenvolvedores de aplicações
        \item Menor portabilidade
        \item Poucos sistemas em produção
    \end{enumerate}
\end{enumerate}
\end{frame}

\begin{frame}[fragile]
\frametitle{Comparação entre Tipos de Kernel}

\begin{center}
\begin{tabular}{|l|c|c|c|c|}
\hline
\textbf{Característica} & \textbf{Monolítico} & \textbf{Microkernel} & \textbf{Híbrido} & \textbf{Exokernel} \\
\hline
Tamanho do kernel & Grande & Pequeno & Médio & Mínimo \\
\hline
Desempenho & Alto & Moderado & Alto/Moderado & Potencial alto \\
\hline
Modularidade & Baixa & Alta & Média & Alta \\
\hline
Segurança & Menor & Maior & Média & Variável \\
\hline
Complexidade & Alta & Média & Alta & Baixa (kernel) \\
 & & & & Alta (LibOS) \\
\hline
Exemplos & Linux, & MINIX, & Windows NT, & MIT Exokernel, \\
 & FreeBSD & QNX & macOS & Nemesis \\
\hline
\end{tabular}
\end{center}
\end{frame}

\begin{frame}[fragile]
\frametitle{Interação de um Sistema com SO}

\begin{columns}
    \begin{column}{0.5\textwidth}
        \begin{enumerate}
            \item Comunicação entre aplicações e kernel:
            \begin{enumerate}
                \item Kernel Monolítico: System Calls e Sinais de Interrupções
                \item Microkernel: IPC, Kernel Calls e Sinais de Interrupções
            \end{enumerate}
            \item Diferenças fundamentais:
            \begin{enumerate}
                \item Monolítico: chamadas diretas ao kernel
                \item Microkernel: comunicação via mensagens entre processos
            \end{enumerate}
        \end{enumerate}
    \end{column}
    
    \begin{column}{0.5\textwidth}
        \begin{center}
            \includegraphics[scale=0.3]{example-image}
        \end{center}
    \end{column}
\end{columns}
\end{frame}

\begin{frame}[fragile]
\frametitle{Tendências Atuais em Design de Kernel}

\begin{enumerate}
    \vfill \item Sistemas operacionais modernos tendem a adotar abordagens híbridas
    \vfill \item Linux: kernel monolítico com módulos carregáveis
    \vfill \item Windows: kernel híbrido com componentes em modo kernel e modo usuário
    \vfill \item Sistemas embarcados: microkernels ganham popularidade pela segurança
    \vfill \item Virtualização e containers: novas demandas para design de kernel
    \vfill \item Sistemas de tempo real: foco em previsibilidade e determinismo
\end{enumerate}
\end{frame}

\begin{frame}{Resumo}
\begin{enumerate}
    \vfill \item Sistemas Operacionais são camadas de software que gerenciam recursos de hardware
    \vfill \item Podem ser vistos como Máquinas Virtuais ou Gerenciadores de Recursos
    \vfill \item O Kernel é o núcleo do sistema operacional, executando em modo privilegiado
    \vfill \item Principais tipos de kernel:
    \begin{enumerate}
        \item Monolítico: todo o SO em modo kernel (Linux)
        \item Microkernel: mínimo em modo kernel, serviços em modo usuário (MINIX)
        \item Híbrido: combinação das abordagens anteriores (Windows NT)
        \item Exokernel: apenas proteção e multiplexação de recursos
    \end{enumerate}
    \vfill \item Cada abordagem apresenta vantagens e desvantagens em termos de desempenho, segurança e modularidade
\end{enumerate}
\end{frame}

\section{Referências}
\begin{frame}[t, allowframebreaks]
    \frametitle{References}
    \printbibliography
\end{frame}

\end{document}
