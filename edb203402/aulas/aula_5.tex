% Modelo para apresentação Beamer IFSC
\documentclass{beamer}

\usepackage[utf8]{inputenc}
\usepackage[T1]{fontenc}
\usepackage[english,brazil]{babel}
\usepackage{amsmath, amssymb, graphicx, caption}
\usepackage{beamerthemeIFSC}

\title{EDB18802 - Eletrônica Digital II:}
\subtitle{\LARGE Contadores Digitais}
\author{João Cláudio Elsen Barcellos}
\date{\scriptsize \the\day~de \ifcase\month\or Janeiro\or Fevereiro\or Março\or Abril\or Maio\or Junho\or Julho\or Agosto\or Setembro\or Outubro\or Novembro\or Dezembro\fi~de \the\year}
\institute{Engenheiro Eletricista \\ Universidade Federal de Santa Catarina \\ \url{joaoclaudiobarcellos@gmail.com}}

\begin{document}

\frame{\titlepage}

\begin{frame}{Plano de Aula}
\tableofcontents
\end{frame}

\section{Os Tipos de Contadores}

\begin{frame}{Contadores: Lógica Simples vs Lógica Sincronizada}
\

\begin{itemize}
  \item Circuitos com e sem clock externo.
  \item Aplicações simples podem dispensar sincronismo.
  \item Aplicações complexas exigem controle preciso do tempo.
  \item Uso de sinal de clock vindo de um oscilador externo.
\end{itemize}

\end{frame}


\begin{frame}{Contadores: Lógica Simples vs Lógica Sincronizada}

\begin{table}[h!]
\centering
\caption{Comparação entre Lógica Simples e Lógica Sincronizada}
\resizebox{0.9\textwidth}{!}{ % ajuste 0.9 para a escala desejada
\begin{tabular}{|l|c|c|}
\hline
\textbf{Característica} & \textbf{Lógica Simples (Assíncrona)} & \textbf{Lógica Sincronizada (Síncrona)} \\
\hline
Uso de clock            & Não                                  & Sim                                     \\
\hline
Complexidade            & Baixa                                & Alta                                    \\
\hline
Tempo de resposta       & Imediato                             & Definido pelo clock                     \\
\hline
Controle de temporização & Difícil                              & Preciso                                 \\
\hline
\end{tabular}
}
\end{table}



\end{frame}






\begin{frame}{Classificações dos Contadores}
\begin{itemize}
  \item \textbf{Quanto ao sincronismo:}
  \begin{itemize}
    \item Assíncronos (Ripple Counter)
    \item Síncronos (Clock comum)
  \end{itemize}
  \item \textbf{Quanto ao modo de contagem:}
  \begin{itemize}
    \item Progressivos (UP)
    \item Regressivos (DOWN)
  \end{itemize}
\end{itemize}
\end{frame}

\section{Contadores Assíncronos}

\begin{frame}{Contador Assíncrono com Flip-Flops J-K}
\begin{columns}
\column{0.4\textwidth}
\begin{itemize}
  \item Clock no primeiro estágio.
  \item Saída Q alimenta o próximo estágio.
  \item Frequência é dividida por 2 a cada estágio.
\end{itemize}
\column{0.6\textwidth}
\includegraphics[width=0.95\columnwidth]{figures/contador_ass.png}
\end{columns}
\end{frame}

\begin{frame}{Diagrama de Tempo: Contador Assíncrono 3 bits}
\centering
\includegraphics[width=0.8\columnwidth]{figures/diagrama_tempo.png}
\end{frame}

\begin{frame}{Contador Decrescente (Assíncrono)}
\begin{columns}
\column{0.4\textwidth}
\begin{itemize}
  \item Uso das saídas \texttt{\textbackslash Q}.
  \item Contagem de 7 até 0.
\end{itemize}
\column{0.6\textwidth}
\includegraphics[width=0.85\columnwidth]{figures/cont_decr_ass.png}
\end{columns}
\end{frame}

\begin{frame}{Capacidade Máxima de Contagem}
\begin{itemize}
  \item Fórmula: \( n = 2^x \)
  \item \( x \) = número de flip-flops
  \item Exemplo: 4 flip-flops \(\Rightarrow\) \( 2^4 = 16 \) (0 a 15)
\end{itemize}
\end{frame}

\section{Contagem Programada}

\begin{frame}{Redefinindo o Ciclo de Contagem}
\begin{itemize}
  \item Limitação: contagem natural até potências de 2.
  \item Solução: uso de entradas CLEAR ou PRESET.
\end{itemize}
\end{frame}

\begin{frame}{Contador Programado com Clear}
\begin{columns}
\column{0.4\textwidth}
\begin{itemize}
  \item RESET ao detectar combinação específica.
  \item Uso de porta lógica ligada às saídas.
\end{itemize}
\column{0.6\textwidth}
\includegraphics[width=0.9\columnwidth]{figures/cont_clear.png}
\end{columns}
\end{frame}

\begin{frame}{Contador Programado com Preset}
\centering
\includegraphics[width=0.8\columnwidth]{figures/cont_pres.png}
\end{frame}

\section{Contadores Up/Down}

\begin{frame}{Contador UP/DOWN}
\begin{columns}
\column{0.4\textwidth}
\begin{itemize}
  \item Entrada seletora define o modo de contagem.
  \item Uso de saídas Q ou \textbackslash Q para alterar direção.
\end{itemize}
\column{0.6\textwidth}
\includegraphics[width=0.8\columnwidth]{figures/up_down.png}
\end{columns}
\end{frame}

\section{Contadores Síncronos}

\begin{frame}{Contadores com Clock Único}
\begin{itemize}
  \item Todos os flip-flops recebem o mesmo clock.
  \item Velocidade independe do número de estágios.
\end{itemize}
\centering
\includegraphics[width=0.8\columnwidth]{figures/clock_unico.png}
\end{frame}

\begin{frame}{Problemas e Soluções}
\begin{itemize}
  \item Crescimento de portas lógicas com mais estágios.
  \item Alternativa: topologia ripple carry síncrona.
\end{itemize}
\end{frame}

\section{Contadores Síncronos Programáveis}

\begin{frame}{Programando a Contagem Síncrona}
\begin{itemize}
  \item Reset ao detectar configuração desejada.
  \item Uso de portas AND ou NAND.
\end{itemize}
\centering
\includegraphics[width=0.6\columnwidth]{figures/cont_prog_sinc.png}
\end{frame}

\section{Contadores TTL}

\begin{frame}{CI TTL 7490 – Contador de Década}
\centering
\includegraphics[width=0.4\columnwidth]{figures/7490.png}
\end{frame}

\begin{frame}{Modo BCD – 7490}
\centering
\includegraphics[width=0.4\columnwidth]{figures/bcd_7490.png}
\end{frame}

\begin{frame}{Modo Divisor por 10 – 7490}
\centering
\includegraphics[width=0.4\columnwidth]{figures/div_10.png}
\end{frame}

\begin{frame}{Outros CIs TTL: 7492, 7493, 74190}
\centering
\includegraphics[width=0.5\columnwidth]{figures/7492.png}

\end{frame}

\begin{frame}{Outros CIs TTL: 7492, 7493, 74190}
\centering
\includegraphics[width=0.6\columnwidth]{figures/7473.png}

\end{frame}

\begin{frame}{Outros CIs TTL: 7492, 7493, 74190}
\centering
\includegraphics[width=0.6\columnwidth]{figures/74190.png}

\end{frame}

\section{Contadores e Divisores CMOS}

\begin{frame}{CI CMOS 4017}
\centering
\includegraphics[width=0.8\columnwidth]{figures/4017.png}
\end{frame}

\begin{frame}{CI CMOS 4018}
\centering
\includegraphics[width=0.5\columnwidth]{figures/4018.png}
\end{frame}

\section{Aplicações e Exercício}

\begin{frame}{Aplicações dos Contadores}
\begin{itemize}
  \item Armazenamento e controle de tempo.
  \item Sequenciamento de eventos.
  \item Conversão entre domínios (freq./tensão).
\end{itemize}
\end{frame}

\begin{frame}{Exercício}
\begin{itemize}
  \item Elabore o circuito de um contador programado que conta de 0 a 5.
  \item Esboce o circuito com a porta lógica apropriada para RESET.
\end{itemize}
\end{frame}

\begin{frame}{Prática}
\begin{itemize}
  \item Monte um contador de 3 bits.
  \item Utilizar CI 7476.
\end{itemize}
\end{frame}

\begin{frame}{Prática}
\centering
\includegraphics[width=0.8\columnwidth]{figures/CONT_PRA.png}
\end{frame}

\begin{frame}{Referências}
\begin{itemize}
  \item Newton C. Braga. Curso de Eletrônica – Eletrônica Digital.
  \item Datasheets dos CIs TTL e CMOS mencionados.
\end{itemize}
\end{frame}

\end{document}
