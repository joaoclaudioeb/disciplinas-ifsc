\documentclass[letterpaper, 11pt]{extarticle}
\usepackage{ifxetex}
\ifxetex
    \usepackage{fontspec}
    \defaultfontfeatures{Ligatures={TeX}}
    \setmainfont{Arial}
\else % pdflatex
    \usepackage{lmodern}            
    \renewcommand{\familydefault}{\sfdefault}
    \usepackage[T1]{fontenc}       
    \usepackage[utf8]{inputenc}
\fi

% ==================================================
\usepackage{tikz}
\usetikzlibrary{positioning}

\usepackage{pgfplots}
\pgfplotsset{compat=1.18}

\usepackage{siunitx}

\usepackage{tikz-timing}
\usetikztiminglibrary[rising arrows]{clockarrows}
\NewDocumentCommand{\busref}{som}{\texttt{%
#3%
\IfValueTF{#2}{[#2]}{}%
\IfBooleanTF{#1}{\#}{}%
}}

% Miscellaneous packages
\usepackage{float}
\usepackage{tabularx}
\usepackage{xcolor}
\usepackage{multicol}
\usepackage{subcaption}
\usepackage{caption}
\captionsetup{format = hang, margin = 10pt, font = small, labelfont = bf}

% Packages for math
\usepackage{mathrsfs}
\usepackage{amsfonts}
\usepackage{amsmath}
\usepackage{amsthm}
\usepackage{amssymb}
\usepackage{physics}
\usepackage{dsfont}
\usepackage{esint}

% ==================================================

% Packages for writing
\usepackage{enumerate}
\usepackage[shortlabels]{enumitem}
\usepackage{framed}
\usepackage{csquotes}

% Citation
\usepackage[style=numeric, citestyle=ieee]{biblatex}
\addbibresource{references.bib} % ou o nome que você der ao seu arquivo .bib


% Hyperlinks setup
\usepackage{hyperref}
\definecolor{links}{rgb}{0.36,0.54,0.66}
\hypersetup{
   colorlinks = true,
    linkcolor = black,
     urlcolor = blue,
    citecolor = blue,
    filecolor = blue,
    pdfauthor = {Author},
     pdftitle = {Title},
   pdfsubject = {subject},
  pdfkeywords = {one, two},
  pdfproducer = {LaTeX},
   pdfcreator = {pdfLaTeX},
   }

% Novo comando para definir entre inglês ou português
\newif\ifenglish\englishfalse
\englishtrue
\newcommand{\lang}[2]{\ifenglish#1\else#2\fi}

% ==================================================

% document parameters
% \usepackage[spanish, mexico, es-lcroman]{babel}
\englishfalse
\lang{
\usepackage[english]{babel}
\usepackage[margin = 1in]{geometry}}
{\usepackage[portuguese]{babel}
\usepackage[margin = 1in]{geometry}}

\newcommand{\DAY}{\the\day}
\newcommand{\MONTH}{%
    \ifcase\month
    \or \lang{January}{janeiro}%
    \or \lang{February}{fevereiro}%
    \or \lang{March}{março}%
    \or \lang{April}{abril}%
    \or \lang{May}{maio}%
    \or \lang{June}{junho}%
    \or \lang{July}{julho}%
    \or \lang{August}{agosto}%
    \or \lang{September}{setembro}%
    \or \lang{October}{outubro}%
    \or \lang{November}{novembro}%
    \or \lang{December}{dezembro}%
    \fi}
\newcommand{\YEAR}{\the\year}
